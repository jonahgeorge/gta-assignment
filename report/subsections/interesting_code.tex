\subsection{Interesting Code}

Of all the interesting code in this project, there are two pieces that stand out.
Particularly, the backgrounds job associated with syncing the applications data with the data in the OSU Course Catalog and the code used to model the integer linear program.
First focusing on the background job code, we are using the ActiveJob interface from Ruby on Rails to define a background job that is run by the DelayedJob adapter.
This implements a queue using a table in Postgres through which a secondary "worker" process will pop jobs off and execute them.
The code included below is the "CourseSyncJob", the second of three background job.
It takes all of the departments declared on the OSU Course Catalog and scrapes it to find each of the department's courses.
These courses are compared to the ones already saved in the database.
If a new course is detected, an object is constructed and saved.
If not, the existing object is updated with any new values that came from the Course Catalog.

\begin{lstlisting}
class CoursesSyncJob < ActiveJob::Base
  queue_as :default

  def perform(department_id, cc_department_json)
    cc_department = OsuCcScraper::Department.from_json(cc_department_json)

    cc_courses(cc_department).each do |cc_course|
      course = Course.find_or_create_by(
        department_id: department_id, course_number: cc_course.course_number)
      course.update_attributes(name: cc_course.name)
      course.save

      SectionsSyncJob.perform_later(course.id, cc_course.to_json)
    end
  end

  private

    def cc_courses(cc_department)
      cc_department.courses.select { |c|
        course_whitelist.include?("#{cc_department.subject_code} #{c.course_number}")
      }
    end

    def course_whitelist
      ["CS 225", "CS 261", "CS 261", "CS 271", "CS 290"]
    end
end
\end{lstlisting}

The second piece of interesting code is a class called \textit{IntegerLinearProgram}.
This class encapsulates all of the logic necessary to model the ILP.
It uses a third-party library called "rulp" to handle the actual Ruby domain-specific language (DSL) to linear-program (LP) modelling language conversion.
In this case, the class constructor takes two ActiveRecord (Ruby on Rails ORM) relations, "students" and "sections" and will query for additional data off of them.
Using this method, we can pass a subset of the students and sections in the system into the IntegerLinearProgram class and treat it as a "blackbox" solution as it will fetch whatever other data it requires.
After constructing all of the constraints in the class constructor, the programmer can call "solve" on the object which serializes the linear program to a string which is passed into the COIN-OR CBC solver.
The results is passed back to Ruby where it is re-serialized into Ruby objects that we can use in other parts of the system.

Because of the size of this class, it has been shortened to show the structure and we'll descend into a few particularly interesting methods. The structure of \textit{IntegerLinearProgram} is as follows:

\begin{lstlisting}
class IntegerLinearProgram

  def initialize(students, sections, fixed_assignments = {}, fte_per_section = 0.25, students_per_ta = 30)
    @students = students
    @sections = sections
    @fixed_assignments = fixed_assignments
    @fte_per_section = fte_per_section
    @students_per_ta = students_per_ta

    @problem = Rulp::Max(objective)
    enrollment_contraint
    fte_contraint
    skill_contraint
    fixed_assignments_constraint if @fixed_assignments
  end

  def solve; end
  def results; end
  def print_results; end

private

  def objective; end
  def enrollment_contraint; end
  def fte_contraint; end
  def skill_contraint; end
  def fixed_assignments_constraint; end

end
\end{lstlisting}

The first interesting method is the \textit{objective} method.
This method is responsible for modelling the objective function of the linear program.
It starts by iterating all of the sections to build a hashmap of the instructor's preferences of students.
The keys of this hash are a combination of the student's id and the section's id.

Next, it starts iterating student-by-student building the linear program variables.
First, it builds a similar hashmap structure for student preference's of sections.
Then starts iterating section-by-section retrieving the student's preference of the section and the instructor's preference of the student for said section.
If either of these values does not exist, it substitutes a value of \textit{1}.
Finally, it builds an array of these LP variables in the \textit{variables} variable.
At the end, \textit{variables} is summed and returned.

\begin{lstlisting}
def objective
  variables = []

  instructor_student_preferences = {}
  @sections.each do |section|
    if section.instructor
      section.instructor.student_preferences.each do |preference|
        instructor_student_preferences[:"#{preference.student.id}_#{section.id}"] = preference
      end
    end
  end

  @students.each do |student|
    section_preferences =
      student.section_preferences.index_by { |preference| :"#{student.id}_#{preference.section.id}" }

    @sections.each do |section|
      student_score = section_preferences[:"#{student.id}_#{section.id}"].try(:value_raw)
      student_score ||= 1

      instructor_score = instructor_student_preferences[:"#{student.id}_#{section.id}"].try(:value_raw)
      instructor_score ||= 1

      variables << (student_score * instructor_score * VAR_b(student.id, section.id))
    end
  end
  variables.sum
end
\end{lstlisting}

The second piece of interesting code is the \textit{skill\_constraint} method.
This is one of the four constraints that are programmed into our linear program.
This one in particular is responsible for preventing teaching assistants from being assigned to classes that they are not qualified to TA.
Similar to the objective function, it starts iterating student-by-student and building a hashmap of experience records.
Then, it iterates by sections and the requirements for each section to determine which constraints need to be added.
If a section has a particular requirement and the student does not fulfill that requirement, a "hard" constraint settings the value to zero is added to the linear program.

\begin{lstlisting}
def skill_contraint
  @students.each do |student|
    experiences = student.experiences.index_by { |experience| :"#{experience.skill_id}" }

    @sections.each do |section|
      section.course.requirements.each do |requirement|
        exp = experiences[:"#{requirement.skill_id}"]
        if exp == nil || exp[:value] < requirement[:value]
          @problem[ VAR_b(student.id, section.id) <= 0 ] # Hardcode to NO
        end
      end
    end
  end
end
\end{lstlisting}
